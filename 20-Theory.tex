\chapter{Fundamentos teóricos}
\label{chap:theo}

Este capítulo é opcional \textbf{mas altamente recomendado} e, a existir, deve descrever os Fundamentos Teóricos, o que incluirá descrever as tecnologias relevantes, sejam concorrentes ou de fundação.


\section{Linguagem C\#}

\section{Linguagem Python}

\section{API}

\subsection{O que é uma API?}

\section{Unix Socket}

\subsection{O que é uma unix socket?}

\subsection{Como funciona?}


\section{Containers}

\subsection{Tipos de containers}

\subsection{LXD e LXC}

\subsection{Incus}

\section{FireWalls Linux}

\subsection{IPTables}

\subsection{NFTables}

\section*{Sumário}

Ver o \nameref{sec:intro_summary} na página \pageref{sec:intro_summary} para perceber como utilizar esta secção.
