\chapter{Introdução}
\label{chap:introduction}

Este relatório pretende demonstrar o trabalho realizado para este projeto.
O mesmo consiste num mecanismo de controlo de portas TCP/UDP \cite{rfc9293},
\cite{rfc768} dos \textit{containers} da plataforma Forge do IPMAIA.


Pequeno texto introdutório. Ao longo do documento,
não se esqueça de usar citações \cite{ismai_ead} para que 1)
reforce a veracidade dos seus conteúdos e 2) reforce a legitimidade da
originalidade do seu trabalho, demonstrando que, apesar de ter 
consultado outras fontes, não está simplesmente a copiar de forma desonrosa
o conteúdo de terceiros e a passá-lo como seu.

\begin{figure}[ht]
\begin{center}
\includegraphics[width=9cm]{figs/logo_ipmaia_small.png}
\caption{Exemplo de inserção de uma imagem.}
\source{\cite{ismai_ead}}
\label{fig:bookstack}
\end{center}
\end{figure}

\section{Objectivos}
\label{sec:object}

Aqui deve colocar os objectivos. O que se pretende com este trabalho?

Outra citação de exemplo: \cite{rfc4512}

Com este projeto pretende-se, criar uma extensão à plataforma forge do IPMAIA, com o principal objetivo de facilitar a gestão de portas TCP/UDP dos \textit{containers} que são disponibilizados.

Para a concretização deste trabalho são necessários os seguintes objetivos:\\

\begin{itemize}
\item \textit{Daemon} que irá interagir com as \textit{firewall's} dos \textit{containers};
\item Interface \textit{Web} e API para interagir com o Daemon;
\item Sistema de notificações de eventos para Microsoft Teams e Discord.
\end{itemize}

\section{Metodologia}
\label{sec:intro_method}
Que metodologia pretende aplicar? Que métodos científicos e que métricas?

\section{Recursos tecnológicos}
\label{sec:intro_resources}


No que diz respeito a recursos tecnológicos, optou-se por usar as seguintes opções:

\begin{itemize}
\item Linguagem C\# com o \textit{framework .NET Core} 7.0 para a criacção do \textit{daemon} que gere
as portas dos \textit{containers} e \textit{unix socket de servidor};
\item Linguagem Python versão 3.11 a \textit{unix socket} cliente;
\item Linguagem Python versão 3.11 com a biblioteca Flask para a API e sistemas de notificações;
\item Linguagem Python versão 3.11 com a biblioteca Flask interface \textit{Web}, juntamente com HTML, CSS, Javascipt e Bootstrap.
\item Máquina virtual com alpine linux 3.19 e WSL com ubuntu 22.04 para testar o projeto.
\end{itemize}


\section{Cronograma}
\label{sec:intro_chronos}

Auto-explicativo e facultativo para projecto, mas altamente recomendado.


\section{Organização do relatório}
\label{sec:intro_struct}

Descreva aqui a organização do relatório. Como está estruturado o seu documento?

\section*{Sumário}
\label{sec:intro_summary}
Se considerar relevante, faça um pequeno resumo do capítulo. É uma secção opcional que pode ser eliminada, mas altamente recomendada.
\\
\\
%Abaixo deve usar a instrução {\textbackslash}cite\{\} para colocar uma relação das citações utilizadas no texto deste capítulo. Aqui liste apenas as citações, pela ordem em que aparecem no texto -- não coloque mais texto. Exemplo:
\\
\\
%\textbf{Referências relevantes:}  \cite{ismai_ead}, \cite{rfc4512}.