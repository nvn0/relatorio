%PT: eis alguns exemplos de como se inserem entradas de siglas
\newacronym{api}{API}{\emph{Application Programming Interface}}
\newacronym{cdn}{CDN}{\emph{Content Delivery Network}}
\newacronym{http}{HTTP}{\emph{Hypertext Transfer Protocol}}
\newacronym{https}{HTTPS}{\emph{Hypertext Transfer Protocol Secure}}
\newacronym{ieee}{IEEE}{\emph{Institute of Electrical and Electronics Engineers}}
\newacronym{ietf}{IETF}{\emph{Internet Engineering Task Force}}
\newacronym{ip}{IP}{\emph{Internet Protocol}}
\newacronym{json}{JSON}{\emph{JavaScript Object Notation}}
\newacronym{ldap}{LDAP}{\emph{Lightweight Directory Access Protocol}}
\newacronym{lxc}{LXC}{\emph{Linux Containers}}
\newacronym{lxd}{LXD}{\emph{Linux Hypervisor}}
\newacronym{nat}{NAT}{\emph{Network Address Translation}}
\newacronym{rest}{REST}{\emph{Representational State Transfer}}
\newacronym{rfc}{RFC}{\emph{Request For Comments}}
\newacronym{sdk}{SDK}{\emph{Software development kit}}
\newacronym{ssh}{SSH}{\emph{Secure Shell Protocol}}
\newacronym{tcp}{TCP}{\emph{Transmission Control Protocol}}
\newacronym{udp}{UDP}{\emph{User Datagram Protocol}} 
\newacronym{url}{URL}{\emph{Uniform Resource Locator}} 
\newacronym{vm}{VM}{\emph{Virtual Machine}}
\newacronym{wsl}{WSL}{\emph{Windows Subsystem for Linux}}
