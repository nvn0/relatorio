\chapter{Discussão de resultados}
\label{cap:results}
%Este capítulo é opcional e pode ser omitido, mas destina-se a responder às seguintes questões:
%\begin{enumerate}
%    \item Que resultados foram obtidos?
%    \item O que é que se conseguiu com isso?
%\end{enumerate}

Neste projeto, as tarefas principais (Implementação de um daemon de gestão de portas
e respetctivo interface \textit{web} e suporte a notificações), inicialmente 
apresentadas na proposta foram concluídas com sucesso.

No caso do daemon, foi implementado com multi \textit{threading} e com interage com
o LXC através da API do LXD de modo a ter capacidade de executar multiplas tarefass com a melhor
\textit{performance}.

No sistema de notificações, existe a possiblidade de usar Discord, E-mail e teoricamente Microsoft Teams.

Adicionalmente, pode ser usada a REST API para interagir com daemon ("Port Controler")
através da linha de comandos, caso se pretenda assim fazer.

De modo geral, foi concebido diversos componentes funcionais que podem ser integrados no
devlab(plataforma forge) do IPMAIA/UMAIA.