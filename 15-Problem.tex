\chapter{Análise do problema}
\label{cap:problem}


O Forge da Universidade da Maia e IPMaia é um serviço com multiplas funções com 
o objetivo de ajudar os alunos a desenvolver os seus trabalhos.
Uma das suas funções é a disponibilização de \textit{containers} que podem ser usados
para alojar aplicações/programas desenvolvidos pelos alunos.

Estes \textit{containers} que são disponibilizados por norma precisam de 
abrir/fechar portas TCP/UDP para o funcionamento de programas ou para os
utilizadores interagirem com o mesmo, como por exemplo a porta 22 usada para o 
SSH \cite{rfc4253}.

Atualmente, não existe um sistema se controlo de portas ou \textit{firewall}, neste
momento qualquer \textit{container} consegue abrir portas e deixá-las acessiveis 
para qualquer utilizador dentro da rede local do IPMAIA/UMAIA.

Sendo assim seria útil haver uma método mais fácil e cómodo para o administrador do sistema
controlar e decidir, através de uma interface gráfica, que portas os \textit{containers} dos utilizadores deverão expor 
para a rede local.

Da mesma forma, os utilizadores poderiam fazer um pedido de abertura de porta através de uma iinterface gráfica.

\section*{Sumário}

Idem... 