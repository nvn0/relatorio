\chapter{Fundamentos teóricos}
\label{chap:theo}

Este capítulo é opcional \textbf{mas altamente recomendado} e, a existir, deve descrever os Fundamentos Teóricos, o que incluirá descrever as tecnologias relevantes, sejam concorrentes ou de fundação.


\section{Linguagem C\#}

fdgh

\section{Linguagem Python}

fghd

\section{API}

fdhg

\subsection{O que é uma API?}

fdgh

\section{Unix Socket}

dfgh

\subsection{O que é uma unix socket?}

fghdfhg

\subsection{Como funciona?}

fdhg


\section{Containers}

dfghj

\subsection{Tipos de containers}

fdgh

\subsection{LXD e LXC}

fdghj

\subsection{Incus}

dfhg

\section{FireWalls Linux}

O \textit{kernel} Linux possui desde a versão 2.4, um \textit{framework} de filtragem de pacotes 
conhecido por Netfilter \cite{netfilter}. Este é utilizado utilizado para manipular e inspecionar 
pacotes de rede à medida que entram, saem ou atravessam o sistema.

Porem, a forma mais comum de interagir com o Netfilter é com outras ferramentas,
nomeadamente as duas mais conhecidas, o Iptables e o Nftables.

\subsection{Iptables}

O Iptables é a principal ferramenta associada ao Netfilter, que é uma interface
de linha de comando para configurar regras de filtragem de pacotes.
É posivel definir políticas de segurança, encaminhamento de pacotes,
redirecionamento de portas. 

As funcionalidades do Iptables são organizadas em \textit{tables} ou tabelas
sendo as principais as a \textit{Filter Table}, a \textit{Nat Table} e a
\textit{Mangle Table}, como se pode ver na tabela \ref{ipt1}.

\begin{table}
\centering
\begin{tabular}{|c|c|c|}
\hline
\multicolumn{3}{|c|}{Iptables}\\
\hline
\rowcolor{yellow!50}\textbf{Filter Table} & \textbf{Nat Table} & \textbf{Mangle Table}\\
\hline
INPUT CHAIN & OUTPUT CHAIN & INPUT CHAIN\\
\hline
OUTPUT CHAIN & PREROUTING CHAIN & OUTPUT CHAIN\\
\hline
FORWARD CHAIN & POSTROUTING CHAIN & FORWARD CHAIN\\
\hline
- & - & PREROUTING CHAIN\\
\hline
- & - & POSTROUTING CHAIN\\
\hline
\end{tabular}
\caption{Tabelas principais do Iptables.}
\label{ipt1}
\end{table}



As principais funcionalidades de cada Table são:

\begin{itemize}
\item \textbf{\textit{Filter Table}}: Esta tabela é usada para filtrar pacotes com base em 
políticas de filtragem, como permitir, negar ou descartar pacotes;
\item \textbf{\textit{Nat Table}}: Esta tabela é usada para modificar endereços IP e portas
nos cabeçalhos dos pacotes, principalmente para implementar NAT;
\item \textbf{\textit{Mangle Table}}: Esta tabela é usada para alterar pacotes de maneiras
não abrangidas pelas outras tabelas. Isso pode incluir marcação de pacotes para fins especiais;
\end{itemize}



Existe ainda outras duas tabelas secundarias com propositos mais simples, dadas
por \textit{Raw Table} e \textit{Security Table}, como se pode ver na
tabela \ref{ipt2}.


\begin{table}
\centering
\begin{tabular}{|c|c|}
\hline
\multicolumn{2}{|c|}{Iptables}\\
\hline
\rowcolor{yellow!50}\textbf{Raw Table} & \textbf{Security Table}\\
\hline
OUTPUT CHAIN & INPUT CHAIN \\
\hline
PREROUTING CHAIN & OUTPUT CHAIN \\
\hline
- & FORWARD CHAIN \\
\hline
\end{tabular}
\caption{Tabelas secundárias do Iptables.}
\label{ipt2}
\end{table}
    

\begin{itemize}
\item \textbf{\textit{Raw Table}}: Esta tabela é usada rastrear conexões, com um mecanismo de
marcação de pacotes de modo a mostrar conexões ativas. permite também aplicar
exceções antes que as regras de outras tabelas sejam aplicadas;
\item \textbf{\textit{Security Table}}: Serve para interagir com o SELinux;
\end{itemize}

No Iptables uma \textit{table} é um conjunto de \textit{chains} em que cada Chain tem uma função
especifica. As \textit{Chains} mais comuns são:


\begin{itemize}
\item \textbf{\textit{INPUT}}: Aplicada a pacotes destinados ao próprio sistema;
\item \textbf{\textit{OUTPUT}}: Aplicada a pacotes gerados pelo próprio sistema;
\item \textbf{\textit{FORWARD}}: Aplicada a pacotes que são reencaminhados pelo 
sistema, ou seja, não são originados ou destinados ao próprio sistema,
mas estão apenas a passar por ele;
\end{itemize}

O Iptables funciona de forma sequencial, ou seja, quando uma regra é colocada
numa \textit{chain} essa regra tem um numero ou posição. Desta forma, quando um 
pacote está a ser processado por uma \textit{chain} são verificadas as regras 
dessa \textit{chain} desde a primeira até à última, até que uma corresponda a
parametros do pacote e portanto, seja ativada.


Numa regra é possivel especificar parametros como a interface de entrada, 
a interface de saída, IP de destino, IP de origem,o protocolo, a porta de 
origem, porta de destino e ação a tomar na regra (aceitar, rejeitar, descartar).

A seguite tabela \ref{ipt3args} mostra uma lista mais completa de parametros a usar na
criação de regras do IPTables:


\begin{table}
\centering
\begin{tabular}{|c|c|}
\hline
\multicolumn{2}{|c|}{Parametros}\\
\hline
\textbf{Prametro} & \textbf{sintaxe} \\
\hline
table & -t nome da table \\
\hline
posição da regra na chain & \textbf{-I} para o topo e \textbf{-A} para o fundo da lista \\
\hline
Chain & INPUT, OUTPUT, FORWARD, PREROUTING, POSTROUTING \\
\hline
protocolo & \textbf{-p} nome do protocolo  \\
\hline
ip de origem & \textbf{-s} endereço IP  \\
\hline
ip de destino & \textbf{-d} endereço IP  \\
\hline
porta de origem & \textbf{--sport} porta  \\
\hline
porta de destino & \textbf{--dport} porta  \\
\hline
ação & \textbf{-j} ACCEPT/REJECT/DROP  \\
\hline
\end{tabular}
\caption{Parametros possiveis na criação de regras.}
\label{ipt3args}
\end{table}
    
Segue agora um exemplo de uma regra para a tabela \textit{Filter} na 
\textit{INPUT Chain} para negar o tráfego de entrada para a porta 22:

\begin{lstlisting}[language=Bash, caption={exemplo de comando}]
iptabels -I INPUT -p tcp --dport 22 -j DROP
\end{lstlisting}

\textbf{Nota}: Não é necessário especificar no comando a tabela  com o parametro
\textbf{-t} quando se quer adicionar uma regra na \textit{Filter Table}.

\subsection{Nftables}

dsfghsdfh

\section*{Sumário}

dsfg

Ver o \nameref{sec:intro_summary} na página \pageref{sec:intro_summary} para perceber como utilizar esta secção.
