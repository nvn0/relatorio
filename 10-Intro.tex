\chapter{Introdução}
\label{chap:introduction}

Este relatório pretende demonstrar o trabalho realizado para este projeto.
O mesmo consiste num mecanismo de controlo de portas TCP/UDP \cite{rfc9293},
\cite{rfc768} dos \textit{containers} ou do próprio sistema \textit{host}
da plataforma Forge do IPMAIA.

Esta extensão à plataforma Forge, tem 
como principal objetivo de facilitar a gestão as conexões \textit{containers} 
que são disponibilizados, fazendo uso de ima interface \textit{web}, onde o adminitrador da 
plataforma pode receber pedidos para abrir novas conexões. 

Para a concretização deste trabalho são necessários os seguintes objetivos:\\

\begin{itemize}
\item \textit{Daemon} que irá interagir com a \textit{firewall} ou API do sistema de \textit{containers};
\item Interface \textit{Web} e API para interagir com o Daemon;
\item Sistema de notificações de eventos para Microsoft Teams e Discord.
\end{itemize}

\section{Metodologia}
\label{sec:intro_method}
Que metodologia pretende aplicar? Que métodos científicos e que métricas?

\section{Recursos tecnológicos}
\label{sec:intro_resources}


No que diz respeito a recursos tecnológicos, optou-se por usar as seguintes opções:

\begin{itemize}
\item Linguagem C\# com o \textit{framework .NET Core} 7.0 para a criacção do \textit{daemon} que gere
as portas dos \textit{containers} e \textit{unix socket de servidor};
\item Linguagem Python versão 3.11 a \textit{unix socket} cliente;
\item Linguagem Python versão 3.11 com a biblioteca Flask para a API e sistemas de notificações;
\item Linguagem Python versão 3.11 com a biblioteca Flask interface \textit{Web}, juntamente com HTML, CSS, Javascipt e Bootstrap.
\item Máquina virtual com alpine linux 3.19 e WSL com ubuntu 22.04 para testar o projeto.
\end{itemize}


\section{Cronograma}
\label{sec:intro_chronos}

Auto-explicativo e facultativo para projecto, mas altamente recomendado.


\section{Organização do relatório}
\label{sec:intro_struct}

Descreva aqui a organização do relatório. Como está estruturado o seu documento?

\section*{Sumário}
\label{sec:intro_summary}
Se considerar relevante, faça um pequeno resumo do capítulo. É uma secção opcional que pode ser eliminada, mas altamente recomendada.
\\
\\
%Abaixo deve usar a instrução {\textbackslash}cite\{\} para colocar uma relação das citações utilizadas no texto deste capítulo. Aqui liste apenas as citações, pela ordem em que aparecem no texto -- não coloque mais texto. Exemplo:
\\
\\
%\textbf{Referências relevantes:}  \cite{ismai_ead}, \cite{rfc4512}.