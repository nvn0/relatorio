\begin{abstract}
\label{resumo}

Este relatório descreve a descrição da elaboraboração de uma extensão de controlo
de portas TCP/UDP de \textit{containers} que poderá ser integrada na plataforma 
Forge (DevLab do IPMAIA/UMAIA).
Os principais vantagens desta extensão consistem aumentar a segurança através do 
controlo de serviços que são 
disponiblizados pelos \textit{containers}, facilitar as ações do administrador do 
sistema que envolvem o controlo de portas através de uma interface web.
\\
\\
\\
\textbf{Expressões-chave:} Extensão. Controlo de portas TCP/UDP. \textit{Containers}. Segurança.

\end{abstract}

\selectlanguage{english}

\begin{abstract}
\label{abstract}
This report describes the development of an plugin for controlling TCP/UDP 
ports of containers that can be integrated into the Forge platform (DevLab of 
IPMAIA/UMAIA). The main advantages of this extension are increasing security by 
controlling the services provided by the containers, and facilitating the system 
administrator's tasks involving port control through a web interface.
 \\
 \\
 \\
 \textbf{Key-expressions:} \textit{Plugin. Controlling TCP/UDP ports. Containers. Security.}
 
 
\end{abstract}
\selectlanguage{portuges}


% PT: Toda esta secção é facultativa
\chapter*{Declaração de honra}
\label{way_of_the_samurai}
Por este meio se declara que este documento é conteúdo original, produzido pelo identificado como autor, não tendo sido previamente apresentado noutro percurso académico ou unidade curricular de qualquer instituição.

Quaisquer referências a outros trabalhos respeitam rigorosamente os regulamentos e melhores práticas de
atribuição, sendo apropriadamente marcadas no texto e identificadas na secção \emph{\nameref{chapter:refs}} na página \pageref{chapter:refs}, sob as normas \acrshort{ieee}\footnote{\acrshort{ieee} --- \acrlong{ieee}} de referenciação.

O autor igualmente declara não se encontrar em qualquer situação de conflito de interesses efectiva ou potencial, por não haver qualquer envolvimento com elementos passíveis de selecção.

%\nameref{chapter:refs}

% TODO PT: reescrever tudo