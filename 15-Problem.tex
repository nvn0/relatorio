\chapter{Análise do problema}
\label{cap:problem}

Descrever aqui o problema que se pretende resolver.


Olha que giro, uma lista:

\begin{enumerate}
    \item item1;
    \item item2;
    \item item3;
    \item item4.
\end{enumerate}


E uma lista não numerada... nice;

\begin{itemize}
    \item abc;
    \item def.
\end{itemize}

O Forge da Universidade da Maia e IPMaia é um serviço com multiplas funções com 
o objetivo de ajudar os alunos a desenvolver os seus trabalhos.
Uma das suas funções é a disponibilização de \textit{containers} que podem ser usados
para alojar aplicações/programas desenvolvidos pelos alunos.

Estes \textit{containers} que são disponibilizados por norma precisam de 
abrir/fechar portas TCP/UDP para o funcionamento de programas ou para os
utilizadores interagirem com o mesmo, como por exemplo a porta 22 usada para o 
SSH \cite{rfc4253}.

Atualmente, não existe um sistema se controlo de portas ou \textit{firewall}, neste momento qualquer container
consegue abrir portas e deixá-las acessiveis para qualquer utilizador dentro da rede local
do IPMAIA/UMAIA.

Sendo assim seria útil haver uma método mais fácil e cómodo para o administrador de sistema
controlar e decidir que portas os \textit{containers} do utilizadores deverão expor para a rede local.



\section*{Sumário}

Idem... 