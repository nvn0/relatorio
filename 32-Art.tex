\chapter{Estado da Arte}
\label{chap:art}

Neste capítulo de Estado de Arte serão descritos o funcionamento relacionado às conexões
de máquinas virtuais\slash\textit{containers} em fornecedores deste tipo serviço. \\

\section{Conexões das Máquinas Virtuais no Google Cloud}

Nesta secção será abordado o funcionamento das conexões das máquinas virtuais no Google Cloud,
para a descrição desta secção foi consultada a documentação oficial do Google Cloud. \cite{googlecloud} \\

\title*{\textbf{Redes e Sub-redes}}


Cada VM faz parte de uma rede VPC (Google Cloud Virtual Private Cloud), que oferece conectividade com outros produtos do Google Cloud e com a internet. As redes VPC podem ser configuradas em dois modos:

\begin{itemize}
\item Redes em modo automático: Têm uma sub-rede em cada região do Google Cloud, usando intervalos
de endereços IP dentro de 10.128.0.0/9 e suportam apenas IPv4.
\item Redes em modo personalizado: Permitem criar sub-redes manualmente em regiões
selecionadas, usando intervalos de IPv4 e IPv6 especificados.
\end{itemize}



\title*{\textbf{Controladores de Interface de Rede (NICs)}}

Cada VM tem uma interface de rede padrão, mas é possível criar interfaces
adicionais para conectá-la a diferentes redes VPC

\title*{\textbf{Endereços IP}}

As interfaces das VMs recebem endereços IPv4 internos do intervalo primário de 
endereços IPv4 da sub-rede. Opcionalmente, você pode configurar endereços IPv4 externos e, 
se a sub-rede suportar IPv6, também pode configurar endereços IPv6. Estes endereços permitem 
a comunicação com outros recursos do Google Cloud e sistemas externos. \\


\begin{itemize}
\item Endereços IP internos: São locais para a rede VPC, redes VPC pareadas ou 
redes on-premises conectadas via Cloud VPN ou Cloud Interconnect.
\item Endereços IP externos: São endereços IP públicos, roteáveis pela internet, 
podendo ser efêmeros ou estáticos.
\end{itemize}



\title*{\textbf{Regras de Encaminhamento}}

Regras de encaminhamento guiam o tráfego para recursos do Google Cloud com base 
em endereço IP, protocolo e porta, lidando com tráfego interno e externo. Estas regras são 
essenciais para configurações como hospedagem virtual, Cloud VPN, VIPs privados e 
balanceamento de carga. \\




\title*{\textbf{Regras de Firewall}}

Regras de firewall da VPC gerem conexões permitidas ou negadas para VMs com 
base em configurações especificadas. Estas regras são sempre aplicadas pelo Google 
Cloud, protegendo as VMs independentemente do seu estado. Por padrão, as redes VPC 
bloqueiam todas as conexões de entrada e permitem todas as conexões de saída. Regras 
personalizadas podem ser criadas para permitir a comunicação interna entre VMs. \\








\section{Conexões das Máquinas Virtuais no Microsoft Azure}

A conectividade das máquinas virtuais (VMs) no Microsoft Azure é gererida por meio 
de uma combinação de redes virtuais (Virtual Networks, ou VNets), sub-redes endereços IP, 
grupos de segurança de rede (NSGs) e gateways.
As descições que se seguem são com base na documentação da Microsoft \cite{azurecloud} \\

\title*{\textbf{Redes Virtuais (VNets)}}

As VNets são redes isoladas no Azure onde você pode implantar recursos do Azure, 
como VMs. Elas permitem a comunicação segura entre diferentes recursos dentro da 
mesma rede e com recursos externos à rede.\\

\title*{\textbf{Sub-redes}}


Dentro de uma VNet, você pode criar sub-redes para segmentar a rede em partes menores. 
Cada VM deve ser implantada numa sub-rede específica. Isto ajuda a organizar e 
isolar recursos de rede. \\


\title*{\textbf{Endereços IP}

\begin{itemize}
    \item IP Privado: Cada VM em uma sub-rede recebe um endereço IP privado para 
    comunicação interna dentro da VNet.
    \item IP Público: Opcionalmente, uma VM pode ter um endereço IP público para 
    ser acessível pela Internet. Esse IP pode ser estático ou dinâmico.
\end{itemize}



\title*{\textbf{Grupos de Segurança de Rede (NSGs)}

Os NSGs são usados para controlar o tráfego de rede numa VNet. 
É possivel associar NSGs a sub-redes ou interfaces de rede específicas de VMs para 
definir regras de segurança que permitam ou neguem tráfego de rede com base no endereço 
IP, porta e protocolo. \\


\title*{\textbf{Azure bastion}}

O Azure Bastion é um serviço PaaS que proporciona uma conectividade segura e contínua às  
VMs diretamente pelo portal do Azure, sem a necessidade de um endereço IP público para as VMs. 



\section*{Sumário}

Pelos exemplos apresentados, por norma as máquinas virtuais possuem tanto um endereço 
IP público como um privado, tendo sempre um sistema NAT ou sub-rede ao longo da infraestrutura.
É sempre possivel adicionar uma \textit{firewall} no caso do Google Cloud existe uma por padrão.



