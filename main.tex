\documentclass[11pt,a4paper]{report}
\usepackage[utf8]{inputenc}
\usepackage[english,portuges]{babel}
\usepackage{amsmath}
\usepackage{graphicx}
\usepackage{tabularx}
\usepackage{adjustbox}
%\usepackage[colorinlistoftodos]{todonotes}
\usepackage{csquotes}
\usepackage{comment}
\usepackage{imakeidx}
%tabela
\usepackage{multirow}
\usepackage{xcolor}
\usepackage[margin=3cm]{geometry}
\usepackage[hidelinks]{hyperref}
\usepackage[toc,acronym,nopostdot,nonumberlist]{glossaries}
\usepackage{titling}
\usepackage{tikzpagenodes}
\usepackage[ddmmyyyy]{datetime}
\usepackage{setspace}
\usepackage{indentfirst}
\usepackage[toc,page]{appendix}

\usepackage[table]{xcolor}
\usepackage{array}
\renewcommand{\arraystretch}{1.5} %altura da celula

\usepackage{float}

\usepackage[bottom]{footmisc}
%\usepackage{caption}
\usepackage{enumitem}
\usepackage{pgfgantt}
\usepackage{fancyhdr}
\usepackage{lastpage}
\usepackage{listings}
\usepackage{etoolbox}

\setlength {\marginparwidth }{2cm}
\usepackage[colorinlistoftodos,prependcaption,textsize=footnotesize]{todonotes}
% TODO PT: confirmar se isto é preciso \interfootnotelinepenalty=10000
%para definir a localização das tabelas e imagens em modo strict
%\usepackage{placeins}
\setcounter{tocdepth}{4}
\setcounter{secnumdepth}{4}

% ========================================
% PT: personalizações
\usepackage{url}
\usepackage{pdflscape}
\usepackage{rotating}
\usepackage{numprint}

% PT: ordenação das referências
\usepackage[sorting=none]{biblatex}

% PT: formatação das legendas das imagens
\usepackage{caption}
\captionsetup[figure]{labelfont={bf,scriptsize},textfont={scriptsize,it}}
\captionsetup[table]{labelfont={bf,scriptsize},textfont={scriptsize,it}}
\captionsetup[lstlisting]{labelfont={bf,scriptsize},textfont={scriptsize}}
\renewcommand{\lstlistingname}{Excerto}

\newcommand{\source}[1]{\vspace{-6pt}\caption*{\emph{(Fonte: {#1})}}}
\newcommand{\adaptedfrom}[1]{\vspace{-6pt}\caption*{\emph{(Adaptado de {#1})} }}


%-----------------------------------------------------------

\lstset{
   language=csh,
   backgroundcolor=\color{lightgray},
   extendedchars=true,
   basicstyle=\footnotesize\ttfamily,
   showstringspaces=false,
   showspaces=false,
   numbers=left,
   numberstyle=\footnotesize,
   numbersep=9pt,
   tabsize=2,
   breaklines=true,
   showtabs=false,
   captionpos=b
}


\lstset{
   language=Python,
   backgroundcolor=\color{lightgray},
   extendedchars=true,
   basicstyle=\footnotesize\ttfamily,
   showstringspaces=false,
   showspaces=false,
   numbers=left,
   numberstyle=\footnotesize,
   numbersep=9pt,
   tabsize=2,
   breaklines=true,
   showtabs=false,
   captionpos=b
}

\lstset{
   language=Bash,
   backgroundcolor=\color{lightgray},
   extendedchars=true,
   basicstyle=\footnotesize\ttfamily,
   showstringspaces=false,
   showspaces=false,
   numbers=left,
   numberstyle=\footnotesize,
   numbersep=9pt,
   tabsize=2,
   breaklines=true,
   showtabs=false,
   captionpos=b
}
%------------------------------------------------------------


% PT: caracteres
\usepackage{pifont}
\usepackage{newunicodechar}
\newunicodechar{✔}{\ding{51}}
%\DeclareUnicodeCharacter{2713}{✔}

\addbibresource{Z0-ref.bib}
%PT: eis alguns exemplos de como se inserem entradas de siglas
\newacronym{api}{API}{\emph{Application Programming Interface}}
\newacronym{http}{HTTP}{\emph{Hypertext Transfer Protocol}}
\newacronym{https}{HTTPS}{\emph{Hypertext Transfer Protocol Secure}}
\newacronym{ieee}{IEEE}{\emph{Institute of Electrical and Electronics Engineers}}
\newacronym{ietf}{IETF}{\emph{Internet Engineering Task Force}}
\newacronym{ip}{IP}{\emph{Internet Protocol}}
\newacronym{json}{JSON}{\emph{JavaScript Object Notation}}
\newacronym{ldap}{LDAP}{\emph{Lightweight Directory Access Protocol}}
\newacronym{lxc}{LXC}{\emph{Linux Containers}}
\newacronym{lxd}{LXD}{\emph{Linux Hypervisor}}
\newacronym{nat}{NAT}{\emph{Network Address Translation}}
\newacronym{rfc}{RFC}{\emph{Request For Comments}}
\newacronym{sdk}{SDK}{\emph{Software development kit}}
\newacronym{ssh}{SSH}{\emph{Secure Shell Protocol}}
\newacronym{tcp}{TCP}{\emph{Transmission Control Protocol}}
\newacronym{udp}{UDP}{\emph{User Datagram Protocol}} 
\newacronym{url}{URL}{\emph{Uniform Resource Locator}} 
\newacronym{vm}{VM}{\emph{Virtual Machine}}
\newacronym{wsl}{WSL}{\emph{Windows Subsystem for Linux}}

\makeglossaries

%colocar aqui variáveis que serão utilizadas no texto:
\newcommand\kbps{\text{\textit{k}bit/s}}

\newenvironment{rotatepage}%
    {\clearpage\pagebreak[4]\global\pdfpageattr\expandafter{\the\pdfpageattr/Rotate 90}}%
    {\clearpage\pagebreak[4]\global\pdfpageattr\expandafter{\the\pdfpageattr/Rotate 0}}%

\newcommand{\specialcell}[2][c]{%
  \begin{tabular}[#1]{@{}c@{}}#2\end{tabular}}

\newdateformat{daymonthyear}{\THEDAY\ de \monthname[\THEMONTH], \THEYEAR}
\onehalfspacing
\parskip=5pt plus 1pt

\hyphenation{ISMAI}
\hyphenation{IPMAIA}
\hyphenation{UMAIA}
\hyphenation{Redmine}
\hyphenation{Subversion}
\hyphenation{Maieutica}
\hyphenation{DevLab}
\hyphenation{Relab}
\hyphenation{COVID}

\begin{document}

\begin{titlepage}
\doublespacing

\begin{tikzpicture}[remember picture,overlay,shift={(current page.center)}]
\node[anchor=center,xshift=0cm,yshift=9cm]{\includegraphics[scale=4]{figs/logo_ipmaia_small.png}};
\end{tikzpicture}

\centering
\vspace{5cm}
\huge Projeto Plugin para o Forge do IPMAIA\\
\vspace{1cm}
\large Plugin de controlo de portas de containers\\
\vspace{2cm}

\Large Nuno Cardoso\\Aluno A036785\\2023/2024\\
\vspace{2cm}
\large Orientação de\\
\large Cláudia Freitas\\
\vspace{1cm}
\normalsize Tecnologias de Informação, Web e Multimédia\\do\\
Instituto Politécnico da Maia\\
\vspace{1cm}
Maia, \daymonthyear\today \\
\vspace{1cm}
%\includegraphics[scale=0.4]{figs/x.png}
%December, 2019
%\large v2.3
\end{titlepage}

\pagenumbering{roman}
\label{ficha}

%Aqui deve ser colocada a ficha de caracterização.
\begin{abstract}
\label{resumo}

Este relatório descreve ...
\\
\\
\\
\textbf{Expressões-chave:} ....

\end{abstract}

\selectlanguage{english}

\begin{abstract}
\label{abstract}
This report describes ....
 \\
 \\
 \\
 \textbf{Key-expressions:} ...
 
 
\end{abstract}
\selectlanguage{portuges}


% PT: Toda esta secção é facultativa
\chapter*{Declaração de honra}
\label{way_of_the_samurai}
Por este meio se declara que este documento é conteúdo original, produzido pelo identificado como autor, não tendo sido previamente apresentado noutro percurso académico ou unidade curricular de qualquer instituição.

Quaisquer referências a outros trabalhos respeitam rigorosamente os regulamentos e melhores práticas de
atribuição, sendo apropriadamente marcadas no texto e identificadas na secção \emph{\nameref{chapter:refs}} na página \pageref{chapter:refs}, sob as normas \acrshort{ieee}\footnote{\acrshort{ieee} --- \acrlong{ieee}} de referenciação.

O autor igualmente declara não se encontrar em qualquer situação de conflito de interesses efectiva ou potencial, por não haver qualquer envolvimento com elementos passíveis de selecção.

%\nameref{chapter:refs}

% TODO PT: reescrever tudo
\chapter*{Agradecimentos}

%Aqui, se assim for intenção do autor, se deverá colocar o texto com os agradecimentos.

Ao longo da meu percurso académico no IPMAIA e realização deste relatório houveram várias pessoas
pelas quais tenho de agradecer.

Gostaria então, de agradecer ao meu pai e a minha avó por me darem a possibilidade de 
estudar no IPMAIA.

Agradecer à professora e orientadora Cláudia Freitas e professor 
João Paredes pelos conhecimentos fornecidos nas displinas que lecionaram e orientação
neste projeto. 

Agradecer ao meu colega e amigo Diogo Pinto pelas dicas e sugestões relacionadas 
ao projeto.

Agradecer aos meus colegas te turma Rui, João, Diogo pelas dicas relacionadas ao
relatório.
 % PT: este capítulo é opcional
%insert index
\tableofcontents

\phantomsection
\addcontentsline{toc}{chapter}{\listtablename} % PT: lista de tabelas. Não modificar
\listoftables

\phantomsection
\addcontentsline{toc}{chapter}{\listfigurename}	% PT: lista de figuras. Não modificar
\listoffigures

\phantomsection
\renewcommand{\lstlistlistingname}{Lista de excertos de código e terminal}
\addcontentsline{toc}{chapter}{\lstlistlistingname}	% The List of Figures (Do not modify)
\lstlistoflistings
\glsaddall
\printglossary[type=\acronymtype]
\printglossary[type=main,title={Glossário},toctitle={Glossário}]
\pagenumbering{arabic}

\chapter{Introdução}
\label{chap:introduction}

Este relatório pretende demonstrar o trabalho realizado para este projeto.
O mesmo consiste num mecanismo de controlo de portas TCP/UDP \cite{rfc9293},
\cite{rfc768} dos \textit{containers} ou do próprio sistema \textit{host}
da plataforma Forge do IPMAIA.

Esta extensão à plataforma Forge, tem 
como principal objetivo de facilitar a gestão as conexões \textit{containers} 
que são disponibilizados, fazendo uso de ima interface \textit{web}, onde o adminitrador da 
plataforma pode receber pedidos para abrir novas conexões. 

Para a concretização deste trabalho são necessários os seguintes objetivos:\\

\begin{itemize}
\item \textit{Daemon} que irá interagir com a \textit{firewall} ou API do sistema de \textit{containers};
\item Interface \textit{Web} e API para interagir com o Daemon;
\item Sistema de notificações de eventos para Microsoft Teams e Discord.
\end{itemize}


\section{Objetivos}
\label{sec:object}

Pretende-se, com este projecto, acrescentar funcionalidades/facilitar tarefas na plataforma ao
DevLab/Forge do IPMAIA/UMAIA que permitam:

\begin{itemize}
\item o administrador do sistema facilmente defenir regras de \textit{firewall} no sistema host da plataforma (através do terminal ou de uma interface web);
\item a atribuíção de endereços IP a \textit{containers} hospedados na plataforma, de modo,
a dar conectividade aos mesmos (através do terminal ou de uma interface web);
\item controlar quais portas dos \textit{containers} são disponiibilizadas para a rede interna da intituição;
\item os alunos, aos quais, lhes foram atribuídos um \textit{container}, fazer um pedido ao administrador do sistema para poder
obter ou não conectividade em determinada porta TCP/UDP;
\item os alunos que efetuaram um pedido de abertura de porta, receber notificação (por Microsoft Teams ou Discord) quando O
pedido for aceite ou negado.

\end{itemize}

Com a enumeração dos objetivos referidos, serão demonstrados ao longo deste documento
todo o proceso de desnvolvimentos dos mesmos.




\section{Metodologia}
\label{sec:intro_method}
Que metodologia pretende aplicar? Que métodos científicos e que métricas?

\section{Recursos tecnológicos}
\label{sec:intro_resources}


No que diz respeito a recursos tecnológicos, optou-se por usar as seguintes opções:

\begin{itemize}
\item Linguagem C\# com o \textit{framework .NET Core} 7.0 para a criacção do \textit{daemon} que gere
as portas dos \textit{containers} e \textit{unix socket de servidor};
\item Linguagem Python versão 3.11 a \textit{unix socket} cliente;
\item Linguagem Python versão 3.11 com a biblioteca Flask para a API e sistemas de notificações;
\item Linguagem Python versão 3.11 com a biblioteca Flask interface \textit{Web}, juntamente com HTML, CSS, Javascipt e Bootstrap.
\item Máquina virtual com alpine linux 3.19 e WSL com ubuntu 22.04 para testar o projeto.
\end{itemize}


\section{Cronograma}
\label{sec:intro_chronos}

\begin{figure}[H]
\begin{center}
\includegraphics[width=16cm]{figs/cronograma.png}
\caption{Cronograma do projeto}
\label{fig:bookstack}
\end{center}
\end{figure}

\section{Organização do relatório}
\label{sec:intro_struct}

Descreva aqui a organização do relatório. Como está estruturado o seu documento?

\section*{Sumário}
\label{sec:intro_summary}
Se considerar relevante, faça um pequeno resumo do capítulo. É uma secção opcional que pode ser eliminada, mas altamente recomendada.
\\
\\
%Abaixo deve usar a instrução {\textbackslash}cite\{\} para colocar uma relação das citações utilizadas no texto deste capítulo. Aqui liste apenas as citações, pela ordem em que aparecem no texto -- não coloque mais texto. Exemplo:
\\
\\
%\textbf{Referências relevantes:}  \cite{ismai_ead}, \cite{rfc4512}.
\chapter{Análise do problema}
\label{cap:problem}


O Forge da Universidade da Maia e IPMaia é um serviço com multiplas funções com 
o objetivo de ajudar os alunos a desenvolver os seus trabalhos.
Uma das suas funções é a disponibilização de \textit{containers} que podem ser usados
para alojar aplicações/programas desenvolvidos pelos alunos.

Estes \textit{containers} que são disponibilizados por norma precisam de 
abrir/fechar portas TCP/UDP para o funcionamento de programas ou para os
utilizadores interagirem com o mesmo, como por exemplo a porta 22 usada para o 
SSH \cite{rfc4253}.

Atualmente, não existe um sistema se controlo de portas ou \textit{firewall}, neste
momento qualquer \textit{container} consegue abrir portas e deixá-las acessiveis 
para qualquer utilizador dentro da rede local do IPMAIA/UMAIA.

Sendo assim seria útil haver uma método mais fácil e cómodo para o administrador do sistema
controlar e decidir, através de uma interface gráfica, que portas os \textit{containers} dos utilizadores deverão expor 
para a rede local.

Da mesma forma, os utilizadores poderiam fazer um pedido de abertura de porta através de uma iinterface gráfica.

\section*{Sumário}

Idem... 

% PT: opcional
\chapter{Fundamentos teóricos}
\label{chap:theo}

Este capítulo é opcional \textbf{mas altamente recomendado} e, a existir, deve descrever os Fundamentos Teóricos, o que incluirá descrever as tecnologias relevantes, sejam concorrentes ou de fundação.


\section{Linguagem C\#}

fdgh

\section{Linguagem Python}

fghd

\section{API}

fdhg

\subsection{O que é uma API?}

fdgh

\section{Unix Socket}

dfgh

\subsection{O que é uma unix socket?}

fghdfhg

\subsection{Como funciona?}

fdhg


\section{Containers}

dfghj

\subsection{Tipos de containers}

fdgh

\subsection{LXD e LXC}

fdghj

\subsection{Incus}

dfhg

\section{FireWalls Linux}

O \textit{kernel} Linux possui desde a versão 2.4, um \textit{framework} de filtragem de pacotes 
conhecido por Netfilter \cite{netfilter}. Este é utilizado utilizado para manipular e inspecionar 
pacotes de rede à medida que entram, saem ou atravessam o sistema.

Porem, a forma mais comum de interagir com o Netfilter é com outras ferramentas,
nomeadamente as duas mais conhecidas, o Iptables e o Nftables.

\subsection{Iptables}

O Iptables é a principal ferramenta associada ao Netfilter, que é uma interface
de linha de comando para configurar regras de filtragem de pacotes.
É posivel definir políticas de segurança, encaminhamento de pacotes,
redirecionamento de portas. 

As funcionalidades do Iptables são organizadas em \textit{tables} ou tabelas
sendo as principais as a \textit{Filter Table}, a \textit{Nat Table} e a
\textit{Mangle Table}, como se pode ver na tabela \ref{ipt1}.

\begin{table}
\centering
\begin{tabular}{|c|c|c|}
\hline
\multicolumn{3}{|c|}{Iptables}\\
\hline
\rowcolor{yellow!50}\textbf{Filter Table} & \textbf{Nat Table} & \textbf{Mangle Table}\\
\hline
INPUT CHAIN & OUTPUT CHAIN & INPUT CHAIN\\
\hline
OUTPUT CHAIN & PREROUTING CHAIN & OUTPUT CHAIN\\
\hline
FORWARD CHAIN & POSTROUTING CHAIN & FORWARD CHAIN\\
\hline
- & - & PREROUTING CHAIN\\
\hline
- & - & POSTROUTING CHAIN\\
\hline
\end{tabular}
\caption{Tabelas principais do Iptables.}
\label{ipt1}
\end{table}



As principais funcionalidades de cada Table são:

\begin{itemize}
\item \textbf{\textit{Filter Table}}: Esta tabela é usada para filtrar pacotes com base em 
políticas de filtragem, como permitir, negar ou descartar pacotes;
\item \textbf{\textit{Nat Table}}: Esta tabela é usada para modificar endereços IP e portas
nos cabeçalhos dos pacotes, principalmente para implementar NAT;
\item \textbf{\textit{Mangle Table}}: Esta tabela é usada para alterar pacotes de maneiras
não abrangidas pelas outras tabelas. Isso pode incluir marcação de pacotes para fins especiais;
\end{itemize}



Existe ainda outras duas tabelas secundarias com propositos mais simples, dadas
por \textit{Raw Table} e \textit{Security Table}, como se pode ver na
tabela \ref{ipt2}.


\begin{table}
\centering
\begin{tabular}{|c|c|}
\hline
\multicolumn{2}{|c|}{Iptables}\\
\hline
\rowcolor{yellow!50}\textbf{Raw Table} & \textbf{Security Table}\\
\hline
OUTPUT CHAIN & INPUT CHAIN \\
\hline
PREROUTING CHAIN & OUTPUT CHAIN \\
\hline
- & FORWARD CHAIN \\
\hline
\end{tabular}
\caption{Tabelas secundárias do Iptables.}
\label{ipt2}
\end{table}
    

\begin{itemize}
\item \textbf{\textit{Raw Table}}: Esta tabela é usada rastrear conexões, com um mecanismo de
marcação de pacotes de modo a mostrar conexões ativas. permite também aplicar
exceções antes que as regras de outras tabelas sejam aplicadas;
\item \textbf{\textit{Security Table}}: Serve para interagir com o SELinux;
\end{itemize}

No Iptables uma \textit{table} é um conjunto de \textit{chains} em que cada Chain tem uma função
especifica. As \textit{Chains} mais comuns são:


\begin{itemize}
\item \textbf{\textit{INPUT}}: Aplicada a pacotes destinados ao próprio sistema;
\item \textbf{\textit{OUTPUT}}: Aplicada a pacotes gerados pelo próprio sistema;
\item \textbf{\textit{FORWARD}}: Aplicada a pacotes que são reencaminhados pelo 
sistema, ou seja, não são originados ou destinados ao próprio sistema,
mas estão apenas a passar por ele;
\end{itemize}

O Iptables funciona de forma sequencial, ou seja, quando uma regra é colocada
numa \textit{chain} essa regra tem um numero ou posição. Desta forma, quando um 
pacote está a ser processado por uma \textit{chain} são verificadas as regras 
dessa \textit{chain} desde a primeira até à última, até que uma corresponda a
parametros do pacote e portanto, seja ativada.


Numa regra é possivel especificar parametros como a interface de entrada, 
a interface de saída, IP de destino, IP de origem,o protocolo, a porta de 
origem, porta de destino e ação a tomar na regra (aceitar, rejeitar, descartar).

A seguite tabela \ref{ipt3args} mostra uma lista mais completa de parametros a usar na
criação de regras do IPTables:


\begin{table}
\centering
\begin{tabular}{|c|c|}
\hline
\multicolumn{2}{|c|}{Parametros}\\
\hline
\textbf{Prametro} & \textbf{sintaxe} \\
\hline
table & -t nome da table \\
\hline
posição da regra na chain & \textbf{-I} para o topo e \textbf{-A} para o fundo da lista \\
\hline
Chain & INPUT, OUTPUT, FORWARD, PREROUTING, POSTROUTING \\
\hline
protocolo & \textbf{-p} nome do protocolo  \\
\hline
ip de origem & \textbf{-s} endereço IP  \\
\hline
ip de destino & \textbf{-d} endereço IP  \\
\hline
porta de origem & \textbf{--sport} porta  \\
\hline
porta de destino & \textbf{--dport} porta  \\
\hline
ação & \textbf{-j} ACCEPT/REJECT/DROP  \\
\hline
\end{tabular}
\caption{Parametros possiveis na criação de regras.}
\label{ipt3args}
\end{table}
    
Segue agora um exemplo de uma regra para a tabela \textit{Filter} na 
\textit{INPUT Chain} para negar o tráfego de entrada para a porta 22:

\begin{lstlisting}[language=Bash, caption={exemplo de comando}]
iptabels -I INPUT -p tcp --dport 22 -j DROP
\end{lstlisting}

\textbf{Nota}: Não é necessário especificar no comando a tabela  com o parametro
\textbf{-t} quando se quer adicionar uma regra na \textit{Filter Table}.

\subsection{Nftables}

dsfghsdfh

\section*{Sumário}

dsfg

Ver o \nameref{sec:intro_summary} na página \pageref{sec:intro_summary} para perceber como utilizar esta secção.


% PT: opcional
\chapter{Estado da Arte}
\label{chap:art}

Neste capítulo de Estado de Arte serão descritos o funcionamento relacionado às conexões
de máquinas virtuais\slash\textit{containers} em fornecedores deste tipo serviço. \\

\section{Conexões das Máquinas Virtuais no Google Cloud}

Nesta secção será abordado o funcionamento das conexões das máquinas virtuais no Google Cloud,
para a descrição desta secção foi consultada a documentação oficial do Google Cloud. \cite{googlecloud} \\

\title*{\textbf{Redes e Sub-redes}}


Cada VM faz parte de uma rede VPC (Google Cloud Virtual Private Cloud), que oferece conectividade com outros produtos do Google Cloud e com a internet. As redes VPC podem ser configuradas em dois modos:

\begin{itemize}
\item Redes em modo automático: Têm uma sub-rede em cada região do Google Cloud, usando intervalos
de endereços IP dentro de 10.128.0.0/9 e suportam apenas IPv4.
\item Redes em modo personalizado: Permitem criar sub-redes manualmente em regiões
selecionadas, usando intervalos de IPv4 e IPv6 especificados.
\end{itemize}



\title*{\textbf{Controladores de Interface de Rede (NICs)}}

Cada VM tem uma interface de rede padrão, mas é possível criar interfaces
adicionais para conectá-la a diferentes redes VPC

\title*{\textbf{Endereços IP}}

As interfaces das VMs recebem endereços IPv4 internos do intervalo primário de 
endereços IPv4 da sub-rede. Opcionalmente, você pode configurar endereços IPv4 externos e, 
se a sub-rede suportar IPv6, também pode configurar endereços IPv6. Estes endereços permitem 
a comunicação com outros recursos do Google Cloud e sistemas externos. \\


\begin{itemize}
\item Endereços IP internos: São locais para a rede VPC, redes VPC pareadas ou 
redes on-premises conectadas via Cloud VPN ou Cloud Interconnect.
\item Endereços IP externos: São endereços IP públicos, roteáveis pela internet, 
podendo ser efêmeros ou estáticos.
\end{itemize}



\title*{\textbf{Regras de Encaminhamento}}

Regras de encaminhamento guiam o tráfego para recursos do Google Cloud com base 
em endereço IP, protocolo e porta, lidando com tráfego interno e externo. Estas regras são 
essenciais para configurações como hospedagem virtual, Cloud VPN, VIPs privados e 
balanceamento de carga. \\




\title*{\textbf{Regras de Firewall}}

Regras de firewall da VPC gerem conexões permitidas ou negadas para VMs com 
base em configurações especificadas. Estas regras são sempre aplicadas pelo Google 
Cloud, protegendo as VMs independentemente do seu estado. Por padrão, as redes VPC 
bloqueiam todas as conexões de entrada e permitem todas as conexões de saída. Regras 
personalizadas podem ser criadas para permitir a comunicação interna entre VMs. \\








\section{Conexões das Máquinas Virtuais no Microsoft Azure}

A conectividade das máquinas virtuais (VMs) no Microsoft Azure é gererida por meio 
de uma combinação de redes virtuais (Virtual Networks, ou VNets), sub-redes endereços IP, 
grupos de segurança de rede (NSGs) e gateways.
As descições que se seguem são com base na documentação da Microsoft \cite{azurecloud} \\

\title*{\textbf{Redes Virtuais (VNets)}}

As VNets são redes isoladas no Azure onde você pode implantar recursos do Azure, 
como VMs. Elas permitem a comunicação segura entre diferentes recursos dentro da 
mesma rede e com recursos externos à rede.\\

\title*{\textbf{Sub-redes}}


Dentro de uma VNet, você pode criar sub-redes para segmentar a rede em partes menores. 
Cada VM deve ser implantada numa sub-rede específica. Isto ajuda a organizar e 
isolar recursos de rede. \\


\title*{\textbf{Endereços IP}

\begin{itemize}
    \item IP Privado: Cada VM em uma sub-rede recebe um endereço IP privado para 
    comunicação interna dentro da VNet.
    \item IP Público: Opcionalmente, uma VM pode ter um endereço IP público para 
    ser acessível pela Internet. Esse IP pode ser estático ou dinâmico.
\end{itemize}



\title*{\textbf{Grupos de Segurança de Rede (NSGs)}

Os NSGs são usados para controlar o tráfego de rede numa VNet. 
É possivel associar NSGs a sub-redes ou interfaces de rede específicas de VMs para 
definir regras de segurança que permitam ou neguem tráfego de rede com base no endereço 
IP, porta e protocolo. \\


\title*{\textbf{Azure bastion}}

O Azure Bastion é um serviço PaaS que proporciona uma conectividade segura e contínua às  
VMs diretamente pelo portal do Azure, sem a necessidade de um endereço IP público para as VMs. 



\section*{Sumário}

Pelos exemplos apresentados, por norma as máquinas virtuais possuem tanto um endereço 
IP público como um privado, tendo sempre um sistema NAT ou sub-rede ao longo da infraestrutura.
É sempre possivel adicionar uma \textit{firewall} no caso do Google Cloud existe uma por padrão.





% PT: opcional
\chapter{Desenvolvimento do tema}
\label{cap:experiments}

Este capítulo é opcional e, a existir, é aqui que deve descrever como é que o seu projecto evoluiu.

O projeto é constituido em quatro partes:

\begin{itemize}
    \item Ambiente de teste do projeto;
    \item \textit{"Port Controller"} - daemon em C\#;
    \item API e Interface Web;
    \item Sitema de notificações;
\end{itemize}


\section{Ambiente de testes do projeto}

asfd

\subsection{Alpine Linux}

De modo a obter, os melhores resultados possiveis durate os estes e experiencias
ao longo deste projeto é de extrema importancia que o sistema operativo de testes
seja o mesmo da plataforma Forge. Sendo assim foi intalada numa máquina virtual
VMware com o alpine linux 3.19.

Especificações atribuidas à maquina virtual:

\begin{itemize}
    \item 4 GB de RAM;
    \item 3 nucleos de para o CPU;
    \item Placa de rede no modo Bridge;
    \item 40 GB de armazenamento;
\end{itemize}


texttt{apk add --upgrade apk-tools \&\& apk upgrade --available}

\subsection{Configuração do LXD/LXC}

No que diz respeito ao LXD/LXC foram instalados os pacotes lxd e lxd-client.


Foi ativado o serviço do lxd como default com o comando rc-update add lxd default
e em seguida ativado com o rc-service lxd start.

Depois foi feita a configuração iniciual com lxd init com a seguintes opções:

\begin{itemize}
    \item f
\end{itemize}

De seguida foi editada a configuração \textit{default} com o 
comando "texttt{lxc profile edit default}" nela deverá ser garantida o
seguinte:

\begin{lstlisting}[language=csh, caption={edição do perfil padrão}]
  eth0:
    name: eth0
    nictype: bridged
    parent: lxdbr0
    type:nic

\end{lstlisting}

\textbf{Nota:} eth0 é a interface do sistema \textit{host} e lxdbr0 é a interface
de rede criada automaticamente pelo LXD.

Com esta configuração os \textit{containers} irão funcionar em modo NAT.


Após isto foram criados dois containers Alpine linux com o seguinte 
comando \texttt{lxc launch images:alpine/3.19 <nome do container>}

por imagem dos containers

\subsection{Testes de FireWall e conexões}

Uma vez os \textit estarem configurados em modo NAT seria necessário criar regras 
de redirecionamento de portas ou ips de modo a os \textit{containers} serem acedidos
por outros dispositivos na rede local.


De modo a realizar testes de conexão foram instalados dentro dos containers
o netcat e o python.

\textbf{Nota:} eth0 é a placa de rede do sistema host e tem o ip local de 192.168.1.80.


Durante os testes realizados, concluiu-se que existem três medodos de criar conectividade:

\textbf{Método 1: Criar regras no lxc network e defenir portas de redirecionamento}

\textbf{Nota:} Neste metodo será usado como exemplo o \textit{container} "c2" que
dentro da máquina virtual host tem o IP 10.195.171.166.

\textbf{Nota nº2:} A máquina virtual \textit{Host} tem o endereço IP 192.168.1.80.

Este metodo usa o comando \texttt{lxc network forward create lxdbr0 192.168.1.80}
para criar a regra e \texttt{lxc network forward port add lxdbr0 192.168.1.80 tcp 7000 10.195.171.205 8000}
para adicionar as portas a serem reencaminhadas, neste exemplo o trafego vindo da
rede local que tenta aceder ao IP 192.168.1.80 (\textit{host}) pela porta 7000 será redirecionado
para a porta 8000 do IP 10.195.171.205 que pertence ao container "c2".

Nesta experiência o container está a executar um servidor http na porta 8000
com o comando \texttt{python3 -m http.server}


\textbf{Nota:} Porta 8000 é a padrão quando não é especificada outra no final do comando.


Foi possivel aceder ao servidor http pelo browser num computador da rede local.
Foram também usados alguns comandos do netcat como o \texttt{nc -w1 -vz 192.168.1.80 7000}.

-> por imagem do teste

Foram usados tambem usados os comandos \texttt{nc 192.168.1.80 7000} e \texttt{nc -l 8000}
de modo a criar uma ligação TCP. Foi também testado a ligação reversa
onde o \textit{container} se tenta ligar ao computador da rede, invertendo os comandos
\texttt{nc 192.168.1.68 3000} e \texttt{nc -l 3000}, ambos os testes foram bem
sucedidos.

\textbf{Método 2: Usar um ip para redirecionar o tráfego para o container}

fgh

\textbf{Método 3: Usar o Iptables para criar portas de redirecionamento}




\texttt{nc ip porta}
\texttt{nc -l porta}
\texttt{nc -w1 -vz ip porta}
\texttt{python3 -m http.server}


sdf


\subsection{intalação do .NET Core}

dsfg

\section{Port Controller}

\subsection{Escolha da tecnologia}

Uma vez que a plataforma forge funciona no sistema \textit{Alpine Linux} seria
necessário escolher ferramentas suportadas por esta distribuição.

No caso desta parte do projeto, foi escolhida a linguagem C\# com o 
\textit{framework} .NET Core versão 7.0.



\subsection{Funcionalidades}

"Em sistemas operativos de computador multitarefa, um daemon é um programa de 
computador executado como um processo em segundo plano, em vez de estar sob o 
controle direto de um utilizador interativo. Tradicionalmente, os nomes dos processos
de um daemon terminam com a letra d, para esclarecer que o processo é de fato um 
daemon e para diferenciar entre um daemon e um programa de computador normal." \cite{daemon}


O \textit{"Port Controller"} tem como principal função estar à escuta de pedidos
do administrador e executar estes pedidos na \textit{FireWall} do sistema
\textit{host} de modo a gerir conexões relacionadas com os \textit{containers}
ativos. Os pedidos são recebidos na componente da \textit{unix socket} em formato
\textit{JSON}.


O \textit{"Port Controller"} é também capaz de interagir com \textit{containers}
do tipo LXD, e Incus e executar comandos dentro destes para interagir com a
\textit{FireWall} Iptables ou Nftables.

\subsection{Estrutura do código}

O código do "Port Controller" é constituido pelos seguintes ficheiros:

\begin{itemize}
    \item Program.cs;
    \item SocketData.cs;
    \item Containers.cs;
    \item Lxc.cs;
    \item Incus.cs;
\end{itemize}

\subsection{Funcionamento do código}


No Ficheiro Program na função \textit{main} está defenido o caminho do ficheiro 
da unix socket que permite a comunição entre o \textit{"Port Controller"} e 
a interface \textit{Web} ou API.


\begin{lstlisting}[language=csh, caption={teste}]
// Caminho do ficheiro do socket
string socketPath = "/tmp/socket_proj";

if (File.Exists(socketPath))
{
    Console.WriteLine("O ficheiro do socket ja existe. A criar um novo...");
    File.Delete(socketPath);
}

\end{lstlisting}




As classes Lxc e Incus recebem herença da classe Containers.


\section{API e Interface Web}

sdfg

\subsection{Estrutura da API}

sdfg

\subsection{Formato do JSON}

sdgf

\section{Sistema de notificações}

sdfg

\subsection{Discord}

fhgds

\subsection{Microsoft Teams}

dsfg

\section*{Sumário}

Ver o \nameref{sec:intro_summary} página \pageref{sec:intro_summary} para perceber como utilizar esta secção.




% PT: opcional
\chapter{Desvios de Procedimento}
\label{cap:detour}

%Este capítulo é opcional e o seu objectivo é descrever os pontos de discordância entre o planeamento e a implementação: o que é que não correu como estava previsto ao certo? Descrever cada situação.


Neste projeto, as tarefas principais (Implementação de um daemon de gestão de portas
e respetctivo interface \textit{web} e suporte a notificações), inicialmente 
apresentadas na proposta foram concluídas com sucesso.

No caso do daemon, foi implementado com multi \textit{threading} e com interage com
o LXC através da API do LXD de modo a ter capacidade de executar multiplas tarefass com a melhor
\textit{performance}.

No sistema de notificações, existe a possiblidade de usar Discord, E-mail e teoricamente Microsoft Teams.

Adicionalmente, pode ser usada a REST API para interagir com daemon ("Port Controler")
através da linha de comandos, caso se pretenda assim fazer.

De modo geral, foi concebido diversos componentes funcionais que podem ser integrados no
devlab(plataforma forge) do IPMAIA/UMAIA.

% PT: opcional
\chapter{Discussão de resultados}
\label{cap:results}
Este capítulo é opcional e pode ser omitido, mas destina-se a responder às seguintes questões:
\begin{enumerate}
    \item Que resultados foram obtidos?
    \item O que é que se conseguiu com isso?
\end{enumerate}

% PT: opcional
\chapter{Trabalho futuro}
\label{cap:future}

Este capítulo pretende apresentar possiveis melhorias para o projeto.


\section{Implementação das tarefas opcionais}

No que diz respeito a Trabalho futuro, neste projeto seria intressante 
implementar as tarefas opcionais sugeridas na proposta sendo essas:

\begin{itemize}
    \item Implementação de um terminal remoto para os 
    containers num interface web;
    \item Implementação de um painel para gestão dos 
    utilizadores autorizados a aceder a cada 
    container.
\end{itemize} 


\section{Aumentar os meios de notificações}

Outra possibilidade seria aumentar o número de plataformas que podem ser usadas 
para enviar notificações, tais como, matrix, rocket chat ou slack. \\


\section{Implementar uma biblioteca/API para o Iptables}

No que diz respeito a código, mais concretamente no componente do "Port Controler",
seria encontrar uma biblioteca/API fidedigna e testada, capaz de interagir com o Iptables,
com o objetivo de reduzir o consumo de recursos do sistema operativo e melhorar 
a performance deste componente.


\chapter{Conclusões e reflexão crítica}
\label{chap:conclusions}

%Neste capítulo deve apresentar as suas conclusões e realizar uma reflexão crítica do trabalho desenvolvido e da sua participação na empresa.

O desenvolvimento deste projeto foi extremamente intressante de realizar, uma vez, que
existe a possibilidade de o mesmo ser integrado no DevLab. Para além disso, o mesmo foi 
concebido com um conjunto de tecnologias, desde as linguagens de programação, aos protocolos, 
sistemas operativos, que com a sua junção se tornou num desafio capaz de testar e aumentar
os conhecimentos do autor do projeto.






\addcontentsline{toc}{chapter}{Referências}

\printbibliography[title={Referências\label{chapter:refs}}]

\label{lastpage1}

\clearpage

\pagenumbering{arabic}
\renewcommand*{\thepage}{A\arabic{page}}
\appendix
%\begin{appendices}


\begin{comment}
    

\chapter{Especificação de requisitos simplificada}
\label{chap:za}

Este é um exemplo de um anexo do documento. É frequente ser necessário listar um conjunto de requisitos que regem o planeamento do projecto. Se assim for, o ideal será colocá-los aqui.

\end{comment}

\begin{comment}
\chapter{Questionários}
\label{chap:zb}

É muito frequente ser necessário realizar questionários para validar a ideia que está a ser trabalhada, de vários pontos de vista:

\begin{enumerate}
    \item do ponto de vista de como é que o público alvo encara uma necessidade que esta ideia pode suprimir;
    \item do ponto de vista de como é que o público alvo reagiu à possibilidade desta ideia ser posta em prática;
    \item do pontos de vista de como é que o público alvo reagiu à interacção com uma implementação da ideia.
\end{enumerate}

Caso tenha procedido à construção de questionários, use este anexo para listar as perguntas efectuadas, bem como os resultados obtidos. Caso tenha realizado vários questionários diferentes (como os listados na enumeração anterior, por exemplo), separe-os por secções.

\end{comment}
%\end{appendices}
\label{lastpage2}
\end{document}